% Ggf. folgende Zeile auskommentieren, falls der Sperrvermerk gewünscht ist.
\begin{comment}
\chapter*{Sperrvermerk} %*-Variante sorgt dafür, das der Sperrvermerk nicht im Inhaltsverzeichnis auftaucht
gemäß Ziffer 1.1.13 der Anlage 1 zu §§ 3, 4 und 5  der Studien- und Prüfungsordnung für die Bachelorstudiengänge im Studienbereich Technik der Dualen Hochschule Baden-Württemberg vom 29.09.2017 in der Fassung vom 25.07.2018:

Der Inhalt dieser Arbeit darf weder als Ganzes noch in Auszügen Personen außerhalb des Prüfungsprozesses und des Evaluationsverfahrens zugänglich gemacht werden, sofern keine anders lautende Genehmigung vom Dualen Partner vorliegt.\\[4ex]

\city, den \today \\[1ex]

\makebox[7cm]{\rule[-0.2cm]{7cm}{0.5pt}} \hfill \makebox[7cm]{\rule[-0.2cm]{7cm}{0.5pt}} \\ % Die Linien
\makebox[7cm]{\autorone} \hfill \makebox[7cm]{\autortwo} \\[10ex] % Die Namen mittig unter den Linien
\end{comment}
\chapter*{Erklärung} %*-Variante sorgt dafür, dass die Erklärung nicht im Inhaltsverzeichnis auftaucht

gemäß Ziffer 1.1.13 der Anlage 1 zu §§ 3, 4 und 5  der Studien- und Prüfungsordnung für die Bachelorstudiengänge im Studienbereich Technik der Dualen Hochschule Baden-Württemberg vom 29.09.2017 in der Fassung vom 25.07.2018.

Wir versichern hiermit, dass unsere Hausarbeit mit dem Thema:
\begin{comment}
Richtiges Auswählen:
Ich versichere hiermit, dass ich meine Bachelorarbeit (bzw. Projektarbeit oder Studienarbeit bzw. Hausarbeit) mit dem Thema:
\end{comment}
\begin{quote}
	\textit{\titel}\textit{ \untertitel }
\end{quote}

selbstständig verfasst und keine anderen als die angegebenen Quellen und Hilfsmittel benutzt wurden.\\[1ex]

\city, den \today \\[1ex]

\makebox[7cm]{\rule[-0.2cm]{7cm}{0.5pt}}\\
\makebox[7cm]{\autorone}\\[2ex]
\makebox[7cm]{\rule[-0.2cm]{7cm}{0.5pt}}\\
\makebox[7cm]{\autortwo}\\[2ex]
\makebox[7cm]{\rule[-0.2cm]{7cm}{0.5pt}}\\
\makebox[7cm]{\autorthree}\\

\rmfamily

\thispagestyle{empty}

\cleardoublepage

