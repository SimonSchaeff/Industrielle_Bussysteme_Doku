% alle Abkürzungen, die in der Arbeit verwendet werden. Die Alphabetische Sortierung übernimmt Latex. Nachfolgend sind Beispiele genannt, welche nach Bedarf angepasst, gelöscht oder ergänzt werden können.

% Bei den unten stehenden Formelzeichen ist erläutert, wie explizite Sortierschlüssel über den Inhalt der eckigen Klammer angegeben werden.

% Allgemeine Abkürzungen aus Vorlage
\nomenclature{Abb.}{Abbildung}
%\nomenclature{Aufz.}{Aufzählung}
\nomenclature{Tab.}{Tabelle}
\nomenclature{bzw.}{beziehungsweise}
\nomenclature{DHBW}{Duale Hochschule Baden-Württemberg}
%\nomenclature{ebd.}{ebenda}
%\nomenclature{et al.}{at alii}
%\nomenclature{etc.}{et cetera}
%\nomenclature{evtl.}{eventuell}
\nomenclature{f.}{folgende Seite}
\nomenclature{ff.}{fortfolgende Seiten}
%\nomenclature{ggf.}{gegebenenfalls}
%\nomenclature{Hrsg.}{Herausgeber}
%\nomenclature{Tab.}{Tabelle}
%\nomenclature{u. a.}{unter anderem}
%\nomenclature{usw.}{und so weiter}
\nomenclature{vgl.}{vergleiche}
%\nomenclature{z. B.}{zum Beispiel}
%\nomenclature{z. T.}{zum Teil}
\nomenclature{engl.}{englisch}
\nomenclature{B. Eng.}{Bachelor of Engineering}

% Abkurzungen, die in Hausarbeit Automation hinzugefügt worden sind:
\nomenclature{WM}{Warehouse Management}
\nomenclature{ERP}{Enterprise Resource Planning}
\nomenclature{COTS}{Commercial Off-The-Shelf}

% Abkürzungen, die im T3100 Projekt hinzugefügt worden sind:
%\nomenclature{SPS}{Speicherprogrammierbare Steuerung}
%\nomenclature{KOP}{Kontaktorientierte Programmierung}
%\nomenclature{FBS}{Funktionsbausteinsprache}
%\nomenclature{TIA}{Totally Integrated Automation}
%\nomenclature{IoT}{Internet of Things}
%\nomenclature{IT}{Informationstechnologie}

% Abkürzungen, die von mir für das Projekt 2000 hinzugefügt worden sind:
% ======================================================================
%\nomenclature{API}{Application Programming Interface}
%\nomenclature{BMC}{Battery Management Controller}
%\nomenclature{BMS}{Batterymanagementsystem}
%\nomenclature{CMB}{Cell Monitoring Board}
%\nomenclature{CVM}{Cell Voltage Measurement Board}
%\nomenclature{rc}{return code}
%\nomenclature{RTC}{Rational Team Concert}
%\nomenclature{SM}{Sicherheitsmechanismus}
%\nomenclature{ISO SPI}{Isolierte Serielle Periphere Schnittstelle}
%\nomenclature{SW}{Software}
%\nomenclature{BEV}{batterieelektrisch betriebene Fahrzeuge}
%\nomenclature{PHEV}{Plug-in-Hybridfahrzeuge}
%\nomenclature{CI}{Continuous Integration}
%\nomenclature{CD}{Continuous Delivery/Deployment}
%\nomenclature{Pkw}{Personenkraftwagen}
%\nomenclature{SW}{Software}
%\nomenclature{TAP}{Test Access Port}
%\nomenclature{DTAB}{Debug and Test Access Block}
%\nomenclature{JTAG}{Joint Test Action Group} 
%\nomenclature{TDI}{Test Data In}
%\nomenclature{TDO}{Test Data Out}
%\nomenclature{TCK}{Test Clock}
%\nomenclature{TMS}{Test Mode Select}
%\nomenclature{TRST}{Test Reset}
%\nomenclature{LED}{Light Emitting Diode}
%\nomenclature{PC}{Personal Computer}
%\nomenclature{D-SUB}{D-Subminiature}
%\nomenclature{USB}{Universal Serial Bus}
%\nomenclature{CAN}{Controller Area Network}
%\nomenclature{LIN}{Local Interconnect Network}
%\nomenclature{K-Line}{Kommunikationsleitung}
%\nomenclature{SENT}{Single Edge Nibble Transmission}
%\nomenclature{CPU}{Central Processing Unit}
%\nomenclature{HIL}{Hardware-in-the-Loop}
%\nomenclature{UDT}{Utility Debug Trace}
%\nomenclature{ISR}{Interrupt-Service-Routine}
%\nomenclature{CSV}{Comma-separated-values}
%\nomenclature{Autosar}{Automotive open System Architecture}
%\nomenclature{RTE}{Runtime Environment}
%\nomenclature{ECU}{Electronic Control Unit}
%\nomenclature{p}{präemptiv}
%\nomenclature{np}{nicht präemptiv}
%\nomenclature{OS}{Operating System}
%\nomenclature{GEN}{Generation}
%\nomenclature{diff}{Differenz}
%\nomenclature{ID}{Identifikatoren}
%\nomenclature{DBC}{Database Container}
%\nomenclature{CANoe}{CAN Open Environment}



% Scheduling-Algorithmus-Arten
%\nomenclature{FCFS}{First come first served}
%\nomenclature{RR}{Round Robin}
%\nomenclature{SJF}{Shortest Job First}
%\nomenclature{SRT}{Short Remaining Time}
%\nomenclature{HRRN}{Highest Response Ratio Next}

% Softwareentwicklungs Begriffe\item 
%\nomenclature{FIFO}{First in First out}
%\nomenclature{}{}
%\nomenclature{}{}
%\nomenclature{}{}
%\nomenclature{}{}



%\nomenclature{\(Hz\)}{Hertz}
%\nomenclature{\(M\)}{Mega} %Präfix
%\nomenclature{\(s\)}{Sekunde}
%\nomenclature{\(n\)}{nano} %Präfix
%\nomenclature{\(L\)}{Prozessorlast}
%\nomenclature{\(t\)}{Zeit}



% Abkürzungen, die von mir für das Projekt 1000 hinzugefügt worden sind:
\begin{comment}
\nomenclature{DIN}{Deutsche Industrienorm}
\nomenclature{VDE}{Die Abkürzung "VDE" steht für "Verband der Elektrotechnik, Elektronik und Informationstechnik e.V.}
\nomenclature{EN}{Europäische Norm}
\nomenclature{ISO}{Internationale Organisation für Normung}
\nomenclature{DGUV V3}{Deutsche Gesetzliche Unfallversicherung Vorschrift 3}
\nomenclature{VEFK}{verantwortiche Elektrofachkraft}
\nomenclature{GBU}{Gefährdungsbeurteilung}
\nomenclature{EV}{Electric Vehicle (Elektrofahrzeug)}
\nomenclature{EVSE}{Electric Vehicle Supply Equipment (Wallbox)}
\nomenclature{CP}{Control Pilot (Kontroll-/Datenleitung)}
\nomenclature{PP}{Proximity Pilot / Plug Present (Ladekabel-Erkennungs-Kontakt)}
\nomenclature{PE}{Protective Earth (Schutzleiter)}
\nomenclature{PWM}{Pulsweitenmodulation}
\nomenclature{AWG}{American Wire Gauge}
\nomenclature{HV}{High Voltage (Hochspannung)}
\nomenclature{TX}{Torx}
\nomenclature{M3}{Metrisches Gewinde M3}
\nomenclature{zus.}{zusätzlich}
\nomenclature{e. V.}{eingetragener Verein}
%\nomenclature{}{}
\end{comment}

% Dateiendungen aus Vorlage
%\nomenclature{EMF}{Enhanced Metafile}
%\nomenclature{JPG}{Joint Photographic Experts Group}
%\nomenclature{PDF}{Portable Document Format}
%\nomenclature{PNG}{Portable Network Graphics}
%\nomenclature{XML}{Extensible Markup Language}

% Formelzeichen aus Vorlage
%\nomenclature[a]{$a$}{Beschleunigung}
%\nomenclature[F]{$F$}{Kraft}
%\nomenclature[m]{$m$}{Masse}
%\nomenclature[P]{$P$}{Leistung}
%\nomenclature{$U$}{Spannung}
%\nomenclature{$R$}{Widerstand}

% Formelzeichen, die von mir für das Projekt T1000 hinzugefügt worden sind:
\begin{comment}
\nomenclature{$T$}{Periodendauer}
\nomenclature{$D$}{Duty Cycle(Pulsweite)}
\nomenclature{$I$}{Stromstärke}
\nomenclature{$f$}{Frequenz}
%\nomenclature[]{$$}{}
\end{comment}

% Einheiten, die von mir für das Projekt T1000 hinzugefügt worden sind:
\begin{comment}
\nomenclature{$Hz$}{Hertz}
\nomenclature{$s$}{Sekunde}
\nomenclature{$V$}{Volt}
\nomenclature{$A$}{Ampere}
\nomenclature{$m$}{Meter}
\nomenclature{$Nm$}{Newtonmeter}
\nomenclature{$\Omega$}{Ohm}
\nomenclature{$m^2$}{Quadratmeter}
\nomenclature{$^\circ$C}{Grad Celsius}
\nomenclature{m}{milli}
\nomenclature{k}{kilo}
\end{comment}

%----------------------------------------------------------------------------------------